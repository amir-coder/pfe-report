\begin{abstract}
    \thispagestyle{plain}
    \pagenumbering{roman}
    \setcounter{page}{1}
    
    L'aphasie est un trouble de langage qui résulte d'une lésion cérébrale (typiquement suite à un \glsxtrshort{avc}).
    L'aphasie de Broca est une déficience de la production du langage causée par une lésion dans l'aire de Broca,
    une région du lobule frontal gauche du cerveau, responsable de la production de la parole.
    Une personne atteinte d'aphasie de Broca peut avoir des difficultés à articuler les mots et les phrases.
    Cependant, elle peut en général comprendre ce qui est dit.
    L'aphasie de Broca est associé à une diminution de la qualité de vie 
    et à une augmentation du risque de dépression et de tentative de suicide.
   
    La rééducation de la parole est le traitement le plus couramment prescrit aux personnes atteintes d'aphasie de Broca.
    En dépit de son efficacité, 
    la rééducation de la parole est un traitement coûteux en termes de temps, argent et ressources humaines.
    Cela la rend indisponible à un grand nombre de personnes souffrant de l'aphasie de Broca.

    L'utilisation des techniques basées sur le \glsxtrlong{nlp} pour améliorer la qualité de vie de ses individus
    est une voie d'exploration  émergente qui a reçu beaucoup d'attention par les chercheurs pendants les années
    dernières.

    Dans ce mémoire, nous nous intéressons à l'utilisation de la \glsxtrlong{mt} et la \glsxtrlong{asr} pour
    automatiser une partie de la procédure de réhabilitation des personnes touchées par l'aphasie de Broca.
    Dans ce but, nous introduisons l'aphasie de Broca ses causes, 
    ses effets et les problèmes avec les traitements classiques.
    Ensuite, nous menons une étude bibliographique sur les travaux existant sur la \glsxtrlong{mt} et la la \glsxtrlong{asr}.
\end{abstract}

\begin{otherlanguage}{english}
    \begin{abstract}
        \thispagestyle{plain}
        \setcounter{page}{2}
        Abstract
    \end{abstract}
\end{otherlanguage}

\renewcommand{\abstractname}{\RL{ملخص}}
\begin{abstract}
    \thispagestyle{plain}
    \setcounter{page}{3}
    \begin{RLtext}
        ملخص
    \end{RLtext}
\end{abstract}

