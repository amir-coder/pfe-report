\begin{abstract}
    \pagestyle{plain}
    \pagenumbering{roman}
    \setcounter{page}{1}
    
    L'aphasie est un trouble de langage qui résulte d'une lésion cérébrale (typiquement suite à un \glsxtrshort{avc}).
    L'aphasie de Broca est une déficience de la production du langage causée par une lésion dans l'aire de Broca,
    une région du lobe frontal gauche du cerveau, responsable de la production de la parole.
    Une personne atteinte d'aphasie de Broca peut avoir des difficultés à articuler les mots et les phrases.
    Cependant, elle peut en général comprendre ce qui est dit.
    L'aphasie de Broca est associé à une diminution de la qualité de vie 
    et à une augmentation du risque de dépression et de tentative de suicide.
   
    La rééducation de la parole est le traitement le plus couramment prescrit aux personnes atteintes d'aphasie de Broca.
    En dépit de son efficacité, 
    la rééducation de la parole est un traitement coûteux en termes de temps, argent et ressources humaines.
    Cela la rend indisponible à un grand nombre de personnes souffrant de l'aphasie de Broca.

    L'utilisation des techniques basées sur le \glsxtrlong{nlp} pour améliorer la qualité de vie de ses individus
    est une voie d'exploration  émergente qui a reçu beaucoup d'attention par les chercheurs pendants les années
    dernières.

    Dans ce mémoire, nous nous intéressons à l'utilisation de la \glsxtrlong{mt} et la \glsxtrlong{asr} pour
    automatiser une partie de la procédure de réhabilitation des personnes touchées par l'aphasie de Broca.
    Dans ce but, nous introduisons l'aphasie de Broca, ses causes, 
    ses effets et les problèmes avec les traitements classiques.
    Ensuite, nous menons une étude bibliographique sur les travaux existant sur la \glsxtrlong{mt} et la la \glsxtrlong{asr}.
    %%
    \\[2cm]
    %%
    \rule{\linewidth}{1pt}

    \textbf{Mots clés --- } Aphasie de Broca, \Glsxtrlong{ml}, \Glsxtrlong{nlp}, \Glsxtrlong{mt}, \Glsxtrlong{asr}.\\
    \rule{\linewidth}{1pt}
\end{abstract}

\begin{otherlanguage}{english}
    \begin{abstract}
        \thispagestyle{plain}
        \setcounter{page}{2}

        Aphasia is a language disorder caused by brain damage (most commonly a stroke).
        Broca's aphasia is a form of aphasia that impairs language production.
        It is caused by an injury to Broca's Area, an area of the frontal lobe of the brain; responsible for language decoding.
        A person suffering from Broca's aphasia may find it difficult to articulate words and sentences.
        However, they generally can understand what is said to them.
        This form of aphasia is associated with a lower quality of life and a higher risk of depression and suicide.

        Speech therapy is the most commonly prescribed remedy to people with Broca's aphasia.
        Despite its effectiveness, it remains an expensive, time-consuming, and effort-heavy process.
        This makes it inaccessible to a significant number of people with aphasia.

        The use of natural language processing-based techniques to improve these people's quality of life
        is an emerging research avenue that has enjoyed the attention of many researchers in recent years.

        In this thesis, we are interested in the use of machine translation and automatic speech recognition
        to partially automate the rehabilitation of people with aphasia.
        To this end, we introduce aphasia, its causes, consequences, and the problems of classical treatment methods.
        We then undertake a bibliographical study of the existing works pertaining to machine translation and automatic speech recognition.
        %%
        \\ [2cm]
        %%
        \rule{\linewidth}{1pt}

        \textbf{Keywords --- } Broca aphasia, Machine learning, Natural language processing, Machine translation, Automatic speech recognition.\\
        \rule{\linewidth}{1pt}
    \end{abstract}
\end{otherlanguage}

\renewcommand{\abstractname}{\RL{ملخص}}
\begin{abstract}
    \thispagestyle{plain}
    \setcounter{page}{3}
    \begin{RLtext}
        الـحـبـسـة إضـطـرابٌ لـغـوي نـاتـج عن تـلـف فـي الـدمـاغ، غـالـبـا نـتـيـجـة سـكـتـة دمـاغـيـة.
        حـبـسـة بـروكـا حـبـسـة تـنـتـج عـن إصـابـة فـي مـنـطـقـة بـروكـا،
        وهـي مـنـطـقـة فـي الـفـص الجـبـهي الأيـسـر للـدمـاغ تـعـنـى بـإنـتـاج الكـلام.
        قـد يـجـد الـمـصـاب بـحـبـسـة بـروكـا صـعـوبـة فـي تـكـويـن الـجـمـل والـكـلـمـات،
        إلا أنـه عـادة يـفـهـم مـا يـقـال.
        تـرتـبـط هـذه الحـبـسـة بـتـدنـي مـسـوى الـعيـش وارتـفتاع  خـطر الاكـتـئـاب والانـتـحـار.

        عـلاج الـنـطق هـو أكـثـر الـعـلاجـات وصـفـا للمـصـابـيـن بـحـبـسـة بـروكـا.
        رغـم نـجـاعـتـه، فـهو يـظـل مـكـلـفـا للـوقـت والـمـال والـجـهـد،
        مـا يـحـول دون تـوفـره لعـدد كـبـيـر مـمـن يـحـتـاجـونـه.

        تـوظـيـف تـقـنـيـات مـعـالـجـة اللـغـة الـطـبـيـعـيـة لـتـحـسـيـن حـيـاة الـمـصـابـيـن بـحـبـسـة بـروكـا
        مـجـال بـحـث حـظي بـاهـتـمـام الـعـديـد مـن الـبـاحـثـيـن فـي الأعـوام الأخـيـرة.

        فـي هـذه الأطروحـة، نـهـتـم بـاسـتـعـمـال الـتـرجـمـة الآلـية والـتـعرف الآلـي عـلى الكـلام
        لتـأديـة جـزء مـن عـلاج الـنـطق لـحـبـسـة بـروكـا أوتـومـاتـيـكـيـا.
        مـن أجـل ذلك، نـبـدأ بـالـتـعـريـف بـحـبـسـة بـروكـا أسـبـابـا ونـتـائـج،
        ثـم نـتـطرق لـعـيـوب الـعـلاجـات الـمـعـتـادة.
        بـعدهـا نـعرض دراسـة بـيـبـلـيـوغـرافـيـة للأعـمـال الـتـي سـبـق إنـجـازهـا فـي مـجـالـي 
        الـتـرجـمـة الآلـية والـتـعرف الآلـي عـلى الكـلام.
    \end{RLtext}
    \rule{\linewidth}{1pt}
    \begin{RLtext}
        \textbf{الـكـلـمـات الـمـفـتـاحـيـة ـــ } حـبـسـة بـروكـا،
        تـعـلم الآلـة،
        مـعـالـجـة اللـغـة الـطـبـيـعـيـة،
        تـرجـمـة آلـيـة،
        تـعرف آلـي عـلى الكـلام.\\
    \end{RLtext}
    \rule{\linewidth}{1pt}
\end{abstract}

