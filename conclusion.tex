\chapter*{Conclusion générale}
\addcontentsline{toc}{chapter}{Conclusion générale}
\label{chap.general-conclusion}

L'aphasie de Broca est un trouble de communication qui touche une partie grandissante de la population mondiale.
Faisant souvent suite à un \gls{avc}, l'aphasie de Broca affecte le décodage des mots et la production de la parole.
Elle diminue ainsi mesurablement la qualité de vie des personnes qui en sont atteintes%
~\cite{Feigin_Brainin_Norrving_Martins_Sacco_Hacke_Fisher_Pandian_Lindsay_2022,Chapey_2008,Ross_Wertz_2010}.

Le traitement le plus courant de l'aphasie de Broca est la rééducation de la parole par un orthophoniste.
En dépit d'être un traitement efficace, il est gourmand en temps, en argent et en ressources humaines,
ce qui en fait un traitement peu accessible pour la majorité des personnes atteintes.
Ce manque d'accessibilité, combiné à la gravité de certaines conséquences de l'aphasie de Broca,
rend urgent le développement de solutions alternatives%
~\cite{recover,Flowers_Skoretz_Silver_Rochon_Fang_Flamand-Roze_Martino_2016}.

Les méthodes informatiques, notamment les techniques de \gls{ml} et de \gls{nlp},
semblent avoir le potentiel de réduire les coûts matériels et humains associés à la rééducation de la parole.
Elles peuvent ainsi faciliter l'accès au traitement de l'aphasie de Broca%
~\cite{Smaili_Langlois_Pribil_2022,Qin_Lee_Kong_Lin_2022,Misra_Mishra_Gandhi_2022}.

Dans ce projet de fin d'étude, nous avons exploré la possibilité d'utiliser la \gls{mt} et l'\gls{asr},
deux techniques de \gls{nlp} basées sur l'apprentissage \gls{s2s} 
pour aider les personnes atteintes de l'aphasie de Broca.

Nous avons d'abord introduit l'aphasie pour familiariser le lecteur avec
ses causes, sa portée, ses effets, les traitements disponibles et les défis auxquels ils sont confrontés.
Après cela, nous avons présenté la modélisation \gls{s2s}, le cadre général de la \gls{mt} et de l'\gls{asr}.
Nous avons posé le problème et présenté les solutions que nous avons évaluées et comparées.
Le résultat de cette comparaison est une supériorité nette du transformeur sur les autres modèles.
Nous avons alors exploré dans le troisième chapitre les différentes publications qui ont étudié
l'utilisation du transformeur pour la résolution de problèmes de \gls{mt} et d'\gls{asr}.

Ensuite, nous avons présenté la conception d'un système 
qui combine un modèle de traduction avec un modèle d'\gls{asr} pour corriger la parole aphasique.
Conformément à cette conception, la réalisation de ce système a été entamée avec Python et PyTorch.
En conclusion, les résultats de notre travail ont été présentés.
Il s'agit d'un modèle de correction avec une interface web à base de texte pour la partie \gls{mt},
et d'un corpus annoté manuellement pour la partie \gls{asr}.

\section*{Perspectives et horizons de recherche futurs}

En dépit d'être encourageants, les résultats de ce projet de fin d'étude sont loin d'être complets.
Plusieurs pistes de recherche peuvent être explorées pour obtenir de meilleurs résultats.

Parmi les axes d'amélioration les plus prometteurs, nous pouvons citer la taille des corpus d'entraînement.
La collecte de plus de données peut permettre d'entraîner les modèles sans besoin de données synthétiques.
Le remplacement de chatGPT par un modèle moins coûteux comme LLAMA%
~\cite{Touvron_Lavril_Izacard_Martinet_Lachaux_Lacroix_Rozière_Goyal_Hambro_Azhar_etal._2023}
et Alpaca~\cite{Zhang_Han_Zhou_Hu_Yan_Lu_Li_Gao_Qiao_2023}
pour la synthèse des erreurs peut permettre d'utiliser un corpus synthétique de tailles plus importantes.

Une autre piste qui mérite d'être explorée est l'utilisation d'un modèle pré-entraîné comme \gls{bart} ou \gls{gpt}
pour la traduction et Whisper pour l'\gls{asr}.
Ces modèles peuvent être utilisés directement 
ou affinés sur un corpus relativement petit pour avoir de meilleurs résultats. 
