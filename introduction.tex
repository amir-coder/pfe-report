\chapter*{Introduction générale}
\addcontentsline{toc}{chapter}{Introduction générale}
\label{chap.intro}

L'\gls{avc} est une condition médicale dans laquelle 
la circulation du sang dans une région du cerveau s'arrête brusquement~\cite{Larousse}.
La \foreignlanguage{english}{World Stroke Organization} estime que 
plus de 101 millions personnes dans le monde sont les victimes d'un \gls{avc}.
La même organisation indique que 12.2 millions \glspl{avc} ont lieu chaque année.
Ce nombre est susceptible d'augmenter avec le vieillissement de la population mondiale.
Les complications d'un \gls{avc} peuvent être totales ou partielles, temporaires ou permanentes.
Elles peuvent inclure la paralysie, l'amnésie et la difficulté à parler ou à comprendre la parole.
Cette dernière complication est appelée \emph{aphasie}%
~\cite{Feigin_Brainin_Norrving_Martins_Sacco_Hacke_Fisher_Pandian_Lindsay_2022}.


\section*{Contexte}

L'aphasie est un trouble de communication qui complique près d'un tiers des cas d'\glspl{avc}%
~\cite{Flowers_Skoretz_Silver_Rochon_Fang_Flamand-Roze_Martino_2016}.
Il s'agit d'une perte partielle ou totale de la capacité de produire ou de comprendre le langage.
Les cas d'aphasie sont classés en fonction de leur sévérité et des facultés affectées en \emph{syndromes aphasiques}.
L'aphasie de Broca est l'un de ces syndromes aphasiques~\cite{Chapey_2008}.

L'aphasie de Broca est une forme d'aphasie qui affecte la capacité de s'exprimer oralement ou par écrit.
Elle résulte d'une lésion dans l'aire de Broca,
une région du cerveau qui est responsable de la production de la parole.
Les personnes qui souffrent d'une aphasie de Broca ont des difficultés à produire des mots,
mais peuvent comprendre ce qui est dit~\cite{Chapey_2008}.

Il est possible de mitiger les effets de l'aphasie de Broca avec des traitements de réhabilitation.
La majorité de ces traitements sont basés sur la rééducation de la parole.
Cela nécessite plusieurs séances (généralement 1--3 par semaine) avec un orthophoniste.
Le taux de succès de ces traitements après 18 mois est de 24\%%
~\cite{Liu_Huang_Xu_Wu_Tao_Chen_2021,Laska_Hellblom_Murray_Kahan_Von_Arbin_2001,recover}.

\section*{Problématique}

Les difficultés de communication causées par l'aphasie en général 
et l'aphasie de Broca en particulier peuvent être très handicapantes pour la vie quotidienne.
Les activités affectées incluent la communication avec les proches, la participation à des activités sociales,
l'exercice d'un emploi ou même la demande d'aide en cas d'urgence~\cite{Hallowell_2017}.
Par conséquent, la qualité de vie des personnes atteintes de l'aphasie de Broca 
est significativement inférieure à celle des personnes en bonne santé~\cite{Pallavi_Perumal_Krupa_2018,Ross_Wertz_2010}.

Ces problèmes ont un impact négatif sur la santé mentale des personnes atteintes de l'aphasie de Broca.
Elles sont plus susceptibles à l'isolation sociale, à la dépression et aux pensées suicidaires,
des pensées dont la communication est inhibée par l'aphasie et l'isolation sociale%
~\cite{Costanza_et_al._2021,Morrison_2016}.
Le résultat est plus qu'un doublement du taux de mortalité des personnes qui souffrent d'aphasie
par rapport aux cas d'\glspl{avc} sans aphasie (36\% contre 16\% dans les 18 mois suivant l'\gls{avc})%
~\cite{Laska_Hellblom_Murray_Kahan_Von_Arbin_2001}.

De plus, la rééducation de la parole, le traitement de réhabilitation le plus courant et le plus efficace 
pour la plupart des syndromes aphasiques --- dont l'aphasie de Broca ---,
est un processus long et coûteux et fatiguant pour les patients.
Cela en fait un traitement inaccessible pour les personnes qui ont des difficultés économiques ou de mobilité.
Or, les personnes âgées ont souvent de telles difficultés~\cite{Jacobs_Ellis_2021,Liu_Huang_Xu_Wu_Tao_Chen_2021}.

Cette inaccessibilité du traitement de réhabilitation pour un groupe de personnes 
qui inclut une grande partie des personnes qui en ont besoin, 
est inacceptable vu les conséquences souvent catastrophiques de l'aphasie.
Il est urgent de trouver et d'implémenter des solutions qui permettent de faciliter l'accès à la réhabilitation
en réduisant le coût, le besoin du déplacement et la nécessité de supervision professionnelle.

\section*{Objectifs}

L'application des techniques d'\gls{ml} et du \gls{nlp} à cette problématique est une piste de recherche 
qui commence à capturer l'attention des chercheurs%
~\cite{Smaili_Langlois_Pribil_2022,Qin_Lee_Kong_Lin_2022,Misra_Mishra_Gandhi_2022,%
Li_Knopman_Xu_Cohen_Pakhomov_2022,Misra_Mishra_Gandhi_2022}.

L'objectif de ce travail est la création d'un système automatique de réhabilitation de la parole aphasique.
La fonction de ce système est de corriger la parole de son utilisateur, 
c'est--à--dire qu'il prend en entrée la parole éventuellement erronée d'une personne aphasique
et produit en sortie une version corrigé de cette parole 
(en parle de système \foreignlanguage{english}{speech-to-speech}).
Pour le réaliser, nous nous intéressons particulièrement à la \gls{mt} et à l'\gls{asr}.

\section*{Organisation de ce mémoire}

Ce document est la synthèse de notre travail de recherche sur la problématique présentée ci--dessus.
Il est organisé en deux parties.
La première est une étude bibliographique sur la littérature existante relative aux sujets abordés.
La deuxième présente la contribution apportée par notre travail.

\subsection*{\nameref{part.sota}}

Dans cette partie, nous présentons les travaux qui ont été faits dans la direction que nous explorons.
Elle est organisée en trois chapitres :
\begin{enumerate}[label={Chapitre \arabic*}]
    \item \nameref{chap.general-notions} :    
    Dans ce chapitre, nous présentons en général les trois sujets qui sont au cœur de ce travail :
    l'aphasie de Broca, la \gls{mt} et la \gls{asr}.
    Nous commençons par une présentation de l'aphasie de Broca 
    dans laquelle nous décrivons ses symptômes, ses causes et ses conséquences.
    Nous passons ensuite à la réhabilitation de la parole où nous présentons son insuffisance actuelle
    par rapport à l'ampleur et la portée du problème.
    Nous terminons par introduire la \gls{mt} et l'\gls{asr} 
    en présentant leurs définitions et une taxonomie des approches existantes.
    
    \item \nameref{chap.s2s} :
    Ce chapitre sert à familiariser le lecteur avec l'apprentissage \gls{s2s},
    le cadre d'étude général dans lequel s'inscrivent la \gls{mt}, l'\gls{asr} et la majorité des tâches de \gls{nlp}.
    Nous commençons par énoncer le problème, 
    puis nous présentons certaines des architectures neuronales qui ont été appliquées dans son cadre.
    Nous clôturons ce chapitre par une comparaison de ces architectures.
    
    \item \nameref{chap.mt-and-asr} :    
    Ce chapitre part de l'étude générale du chapitre~\ref{chap.s2s}.
    Il détaille l'application du transformeur aux problèmes de \gls{mt} et d'\gls{asr}.
    Nous y présentons les travaux les plus importants dans ces deux domaines.
\end{enumerate}

\subsection*{\nameref{part.contribution}}

Dans cette partie, nous présentons notre contribution à la problématique de la réhabilitation de la parole aphasique.
Elle est aussi organisée en trois chapitres :
\begin{enumerate}[label=Chapitre \arabic*,resume]
    \item \nameref{chap.conception} :
    Le premier chapitre de cette partie présente la conception de notre système.
    Nous commençons par présenter l'architecture générale de notre solution.
    Nous détaillons ensuite la partie \gls{asr} de cette architecture.
    Après cela, la partie \gls{mt} est discutée.
    Pour chaque partie, nous décrivons la procédure d'acquisition de données. 
    Pour la partie \gls{mt}, la création du modèle est aussi présentée.
    
    \item \nameref{chap.realisation} :
    Ce chapitre est consacré à la discussion des choix techniques que nous avons faits.
    Nous présentons d'abord les outils, technologies et bibliothèques que nous avons utilisés.
    Nous détaillons ensuite l'implémentation de la collecte et organisation des données.
    Après cela, la création et l'entraînement du modèle \gls{mt} sont présentés.
    Nous terminons par la présentation de l'interface utilisateur.
    
    \item \nameref{chap.results} :
    Le dernier chapitre de ce mémoire est une discussion des résultats obtenus 
    (notamment dans la partie \gls{mt}).
    D'abord, la qualité et les caractéristiques des données genrées et du corpus créé sont discutées.
    Ensuite, nous présentons et commentons les résultats de l'entraînement.
    Finalement, nous terminons avec les résultats du réglage des hyperparamètres.
\end{enumerate}

