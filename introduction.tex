\chapter*{Introduction générale}
\addcontentsline{toc}{chapter}{Introduction générale}
\label{chap.intro}

L'aphasie est un trouble linguistique qui complique un grand pourcentage d'\glsxtrlongpl{avc},
une condition médicale qui touche plus de 12.2 millions de personnes par an.
Ce chiffre est susceptible d'augmenter avec l'augmentation de l'espérance de vie%
~\cite{Feigin_Brainin_Norrving_Martins_Sacco_Hacke_Fisher_Pandian_Lindsay_2022}.

L'aphasie de Broca est une forme d'aphasie qui affecte la capacité de s'exprimer oralement ou par écrit.
Elle résulte d'une lésion dans l'aire de Broca,
une région du cerveau qui est responsable de la production de la parole.
Les personnes qui ont l'aphasie de Broca ont des difficultés à produire des mots,
mais peuvent comprendre ce qui est dit~\cite{Chapey_2008}.

Ces difficultés peuvent avoir des conséquences néfastes sur plusieurs aspects de la vie quotidienne.
Ceci peut inclure la communication avec les proches,
la participation à des activités sociales,
l'exercice d'un emploi
ou même la demande d'aide en cas d'urgence~\cite{Hallowell_2017}.

L'utilisation de techniques d'\glsxtrlong{ml} et du \glsxtrlong{nlp} 
au bénéfice des personnes atteintes de l'aphasie est une piste de recherche  
qui commence à capturer l'attention de plusieurs chercheurs%
~\cite{Smaili_Langlois_Pribil_2022,Qin_Lee_Kong_Lin_2022,Misra_Mishra_Gandhi_2022}.

Dans ce mémoire, notre objectif est de fournir une revue de l'état de l'art
sur les travaux qui ont été faits dans cette direction.
Nous portons une attention particulière à la \glsxtrlong{mt} et à la \glsxtrlong{asr}
appliquées à l'aphasie de Broca.

Pour ce faire, nous avons organisé notre travail en trois chapitres :

\begin{enumerate}
    \item \nameref{chap.general-notions}.
    
    Dans ce chapitre, nous présentons en général les domaines de recherche qui nous intéressent.
    À cette fin, le chapitre est divisé en trois sections :
    \begin{enumerate}[label=(\arabic*)]
        \item La première section présente l'aphasie de Broca,
        \item la deuxième introduit la traduction automatique
        \item et la troisième présente la reconnaissance automatique de la parole.
    \end{enumerate}

    \item \nameref{chap.s2s}.

    Ce chapitre sert à familiariser le lecteur avec le cadre d'étude général
    dans lequel s'inscrivent la \glsxtrlong{mt}, la \glsxtrlong{asr}
    et la majorité des tâches de \glsxtrlong{nlp}.
    Il s'agit de la modélisation de séquences.
    Nous y présentons l'énoncé du problème et les différentes architectures neuronales
    qui ont été proposées pour le résoudre.

    \item \nameref{chap.mt-and-asr}.
    
    Ce chapitre part de l'étude générale faite dans le chapitre~\ref{chap.s2s}.
    Il détaille l'application des meilleures architectures neuronales
    qui y sont présentées dans le cadre de la \glsxtrlong{mt} et de la \glsxtrlong{asr}.
\end{enumerate}

