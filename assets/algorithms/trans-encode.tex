\RestyleAlgo{ruled}
\begin{algorithm}
    \caption{Encoder une séquence}
    % \label{algo:trans-encode}
    \SetKwFunction{mha}{multi\_headed\_attention}
    % \DontPrintSemicolon
    \(x = (x_1, x_2, \cdots, x_n)\in\reals^n\) \comment{Plongement de la phrase d'entrée.}
    \(N\) \comment{Nombre d'itérations de l'encodeur.}
    \Deb{
        \(z \gets x\)
        \Pour{\(i = 1\cdots N\)}{
            \(z \gets \mha{z} + z\)
        }
    }
    \KwOut{\(z\in\reals^{n\times d}\) \comment{Thought vectors}}
\end{algorithm}