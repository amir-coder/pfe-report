\chapter{Aphasie de Broca}
\newcommand{\greekaphasia}{\textalpha\textphi\textalpha\textsigma\textiota\textalpha }

L'aphasie ; emprunté au Grec ancien ``\greekaphasia'' qui veut dire ``mutisme'',
est un trouble de communication d'origine neurologique~\cite{Larousse}. 
Elle affecte la capacité à comprendre le langage, s'y exprimer ou les deux.
L'aphasie n'est pas causée par un trouble moteur, sensoriel, psychique ou intellectuel~\cite{Chapey_2008},
mais souvent par des \Acrshortpl{avc}, infections ou tumeurs cérébrales, traumatisme crânien, 
troubles métaboliques comme le diabète ou maladies neurodégénératives~\cite{Hallowell_2017}.

\section{Généralités sur le cerveau}

Pour comprendre les causes de l'aphasie, 
il convient de commencer par le cerveau et notamment sa fonction cognitive de communication.