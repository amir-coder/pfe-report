\chapter{Aphasie de Broca}

L'aphasie ; emprunté au Grec ancien \textgreek{>afas'ia} qui veut dire ``mutisme'',
est un trouble de communication d'origine neurologique~\cite{Larousse}. 
Elle affecte la capacité à comprendre le langage, s'y exprimer ou les deux.
L'aphasie n'est pas causée par un trouble moteur, sensoriel, psychique ou intellectuel~\cite{Chapey_2008}.
Sa cause principale est un \Acrshort{avc}, 
mais elle peut également être le résultat d'une infection ou tumeur cérébrale, un traumatisme crânien, 
un trouble métabolique comme le diabète ou une maladie neurodégénérative comme l'Alzheimer ~\cite{Hallowell_2017}.

\subimport{}{history}
\subimport{}{classification}
\subimport{}{brain}