\section{Notes d'histoire}

Louis Victor Leborgne, né en 1809 à Moret-sur-Loing commença à perdre la capacité de parler à l'age de 30 ans.
Il fut admis à l'hôpital de Bicêtre où il passerait 21 ans pendant lesquelles, 
il ne communiquait qu'en produisant le son ``tan'', typiquement répété deux fois, 
si bien qu'on lui a donné le surnom ``monsieur Tan Tan''~\cite{Mohammed_Narayan_Patra_Nanda_2018}.

Le 11 avril 1861, monsieur Leborgne fut examiné par Dr Pierre Paul Broca.
