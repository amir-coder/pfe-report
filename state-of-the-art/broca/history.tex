\section{Notes d'histoire}

Louis Victor Leborgne, né en 1809 à Moret-sur-Loing commença à perdre la capacité de parler à l'age de 30 ans.
Il fut admis à l'hôpital de Bicêtre où il passerait 21 ans pendant lesquelles, 
il ne communiquait qu'en produisant le son ``tan'', typiquement répété deux fois, 
si bien qu'on lui a donné le surnom ``monsieur Tan Tan''~\cite{Mohammed_Narayan_Patra_Nanda_2018}.

Le 11 avril 1861, monsieur Leborgne fut examiné par Dr.~Pierre Paul Broca pour une gangrène dans son pied droit.
Dr.~Broca s'intéressa au trouble linguistique dont souffrait son patient.
Il fit l'observation que les facultés intellectuelles et motrices de monsieur Leborgne étaient intactes,
il en conclut qu'elles ne peuvent être à l'origine de son handicape. 
Broca nomma ``aphémie'' ce type de situation, il en écrivit:

\begin{quotation}
    ``Cette abolition de la parole, chez des individus qui ne sont ni paralysés ni idiots, constitue un symptôme assez singulier pour qu'il me paraisse utile de la désigner sous un nom spécial. Je lui donnerai donc le nom d'aphémie (\textgreek{a} privatif ; \textgreek{fhmi}, je parle, je prononce) ; car ce qui manque à ces malades, c'est seulement la faculté d'articuler les mots.''
    \begin{flushright}
        \rm --- \citeauthor{Broca}, 1861.
    \end{flushright}
\end{quotation}



