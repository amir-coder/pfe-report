\section{\Glsfmtlong{asr}}

L'\gls{asr} est une autre branche du \gls{nlp} 
qui peut faciliter l'informatisation du traitement de l'aphasie de Broca.
Elle permet de convertir la parole du patient en texte qui peut être corrigé avec la \gls{mt}.
Dans cette section, nous la présentons brièvement.


\subsection{Généralités}

La \glsxtrlong{asr} (ou encore la transcription automatique) 
est une technique qui permet de convertir un signal audio en texte.


\subsection{Classification}

Les méthodes d'\gls{asr} peuvent être divisées en méthodes basées sur l'acoustique,
méthodes basées sur la reconnaissance de motifs
et méthodes basées sur l'intelligence artificielle (voir Figure~\ref{fig.asr-taxonomy-tree}).

\begin{figure}
    \centering
    \resizebox{\textwidth}{!}{\begin{tikzpicture}
    \tikzset{every tree node/.style={align=center,anchor=north}}
    \tikzset{level distance=85pt, sibling distance=18pt}
    \Tree 
    [.{\Glsxtrlong{asr}} 
        {Approche acoustique}     
        {Approche de\\reconnaissance\\de motifs} 
        [.{Approche par\\intelligence\\artificielle} 
            {Système à base\\de règles} 
            {Système à base de\\réseaux\\de neurones} 
        ]
    ]
\end{tikzpicture}}
    \caption[Taxonomie des techniques d'\glsfmtshort{asr}.]
    {Taxonomie des techniques d'\glsfmtshort{asr}~\cite{Volny_Novak_Zezula_2012}.}
    \label{fig.asr-taxonomy-tree}
\end{figure}

Comme pour la \gls{mt},
notre intérêt porte principalement sur les méthodes à base d'intelligence artificielle.
Plus spécifiquement sur les méthodes qui utilisent le \gls{dl}.