\section{Solution proposée}

Il n'existe pas de traitement général de l'aphasie de Broca.
L'intervention thérapeutique est donc adaptée à chaque patient~\cite{Acharya_Wroten_2022}.
Un point commun à la plupart des thérapies, 
est l'utilisation d'exercices orthophoniques pour la rééducation de la parole.
Des exemples de tels exercices sont 
la répétition de phrases, la description d'images et la narration de récits~\cite{recover}.
En dépit d'être la méthode la plus efficace dont on dispose,
cette approche est très chronophage étant donné que 
la majorité des patients ne bénéficient que de 1--3 séances par semaine~\cite{recover}.
Elle est également coûteuse, 
avec un coût moyen dans les milliers de dollars par patient~\cite{Liu_Huang_Xu_Wu_Tao_Chen_2021,Jacobs_Ellis_2021},
ce qui la rend très inaccessible.
Ces lacunes sont inacceptables dans le contexte des 
effets dévastateurs de l'aphasie de Broca que nous avons décrits précédemment.

\subimport{}{automatic}