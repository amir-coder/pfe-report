\section{Classification des méthodes de traduction automatique}

La classification des méthodes de \acrshort{ta} la plus citée à travers la littérature, 
repose sur les outils mathématiques de celles-ci.
On distingue notamment trois familles de méthodes~\cite{deep-nmt-survey}:
\begin{enumerate}
    \item Des méthodes basées sur des connaissances linguistiques (règles de traduction).
    \item Des méthodes basées sur les statistiques.
    \item Des méthodes basées sur les réseaux de neurones.
\end{enumerate}

On les appelle respectivement \Acrfull{tabr}, \Acrfull{tas}, \Acrfull{tan}.
Les méthodes dans chacune de ces trois catégories peuvent être encore classifiées~\cite{deep-nmt-survey,hybrid-mt},
ce qui donne lieu à la hiérarchie représentée par la Figure~\ref{fig:mt-taxonomy-tree}.

\begin{figure}
    \begin{center}
       \resizebox{\textwidth}{!}{
        \begin{tikzpicture}
    \tikzset{every tree node/.style={align=center,anchor=north}}
    \tikzset{level distance=80pt, sibling distance=18pt}
    \Tree 
    [.{Traduction automatique} 
        [.{Aproche rationnelle} 
            [.{Méthodes à base de règle} 
                {Méthodes\\dirèctes} 
                {Méthodes\\de transfert} 
                {Méthodes\\par interlingue} 
            ] 
        ] 
        [.{Aproche empirique} 
            [.{Méthodes statistiques} 
                {À base de\\phrases} 
                {Hierarchique\\à base de\\phrases} 
                {À base de\\syntaxe} 
            ] 
            [.{Méthodes neuronales} 
                {Shallow} %todo translate shallow to french
                {TAS avec un\\modèle de langage\\neuronale} 
                {TAN profonde} 
                {TAN à base de\\mécanismes d'attention} 
            ] 
        ] 
    ]
\end{tikzpicture} 
       }
    \end{center}
    
    \caption{Taxonomie des méthodes de traduction automatique.}
    \label{fig:mt-taxonomy-tree}
\end{figure}

Le reste de ce chapitre sera organisé selon la structure donnée par la Figure~\ref{fig:mt-taxonomy-tree}. 
Cela nous donne la structure suivante

\begin{itemize}
    \item \nameref{sec:rbmt}
    \item \nameref{sec:smt}
    \item \nameref{sec:nmt-classic}
    \item \nameref{sec:dnmt}
    \item \nameref{sec:nmt-attention}
\end{itemize}

\subimport{}{rbmt}
\section{Traduction automatique statistique}
\label{sec:smt}
\section{Traduction automatique neuronale classique}
\label{sec:nmt-classic}
\section{Traduction automatique neuronale profonde}
\label{sec:dnmt}
\section{Traduction automatique à base d'attention}
\label{sec:nmt-attention}
