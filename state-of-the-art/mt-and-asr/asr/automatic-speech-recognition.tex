\section{\Glsfmtlong{asr}}
\label{sec.asr}

L'\gls{asr} est un scénario d'apprentissage \gls{s2s} où 
l'ensemble de départ est celui des signaux numériques audio \(s\in\mathbb{R}^\ast\)%
\footnote{%
   On rappelle que \(A^\ast \colonequals \bigcup\limits_{n\in\mathbb{N}} A^n\).
}, %
et l'ensemble d'arrivée est un langage \(L\) sur un vocabulaire \(\Sigma\).
Étant une tâche d'apprentissage \gls{s2s}, l'\gls{asr} se fie très naturellement au traitement par transformeurs.
En effet, il est plus facile de les appliquer à cette tâche, car l'entrée est déjà un vecteur.
Plusieurs travaux ont investigué l'application des transformeurs à l'\gls{asr}.
Deux travaux en particulier ont eu un succès retentissant.
Il s'agit de \cite{Schneider_Baevski_Collobert_Auli_2019} 
et de \cite{Radford_Kim_Xu_Brockman_McLeavey_Sutskever_2022}.

\subimport{}{wav2vec}
\subimport{}{whisper}