\chapter{Traduction automatique et reconnaissance automatique de la parole}
\label{chap.mt-and-asr}

Dans le chapitre~\ref{chap.general-notions}, 
nous avons introduit la \gls{mt} et la \gls{asr} comme avenues possibles 
pour la réhabilitation de la parole chez les patients de l'aphasie de Broca.
Ensuite, dans le chapitre~\ref{chap.s2s}, nous avons présenté le problème général
dont ces deux tâches sont des cas particuliers : celui de la modélisation \gls{s2s}.
Nous y avons posé formellement le problème 
et présenté les architectures neuronales majeures qui ont été utilisées pour le résoudre en les comparant.
Dans ce chapitre, nous abordons dans plus de détails les aspects spécifiques de ces deux tâches.
Nous étudions l'application des architectures présentées (notamment le transformeur) dans leur contexte.

\subimport{mt}{machine-translation}
\subimport{asr}{automatic-speech-recognition}