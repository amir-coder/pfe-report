\section{Traduction automatique à base de règles}
\label{sec:rbmt}

La \acrfull{tabr} est historiquement le premier paradigme de \acrshort{ta}. 
Étant apparue pendant les années 1950s, 
elle resterait l'approche dominante de \acrshort{ta} jusq'aux 1980s~\cite{routledge}.

Comme son nom l'indique, la \acrshort{tabr} est basée sur des règles de traduction explicites,
qui sont généralement créées manuellement à partir de connaissances linguistiques sur la \acrfull{ls} et la \acrfull{lc}. 
Les règles en question peuvent être d'ordre lexical (i.e des dictionnaires),
syntaxique (i.e des grammaires) ou sémantique.

Toutes les méthodes de \acrshort{tabr} passent par deux phases: 
l'analyse de l'entrée dans la \acrshort{ls} 
et la génration (ou synthèse) de la sortie dans la \acrshort{lc}. 
Cependant, les règles utilisées dans ces deux phases peuvent varier en profondeur
et dans le types de connaissances linguistiques employées. 
On distingue ainsi la \acrshort{tabr} en trois sous-familles de méthodes comme indiqué 
sur le sous-arbre gauche de la Figure~\ref{fig:mt-taxonomy-tree}.

\begin{figure}[h]
    \begin{center}
        \begin{tikzpicture}[every edge quotes/.style = {auto, font=\footnotesize, sloped}]
   \node (LS) at (0, 0) {\acrshort{ls}};
   \node (IS) at (1, 1.5) {};
   \node (LC) at (4, 0) {\acrshort{lc}};
   \node (IC) at (3, 1.5) {};
   \node (IL) at (2, 3) {\acrshort{il}};
   \graph {
      (LS) ->["Directe"'] (LC),
      (LS) ->["Analyse"] (IL),
      (IL) ->["Géneration"] (LC),
      (IS) ->["Transfert"'] (IC)
   };
\end{tikzpicture}
    \end{center}
    \caption{Triangle de Vauquois}
    \label{fig:vauquois-triangle}
\end{figure}


