\section{Énoncé du problème}
\label{sec.statement}

Formellement, le problème de modélisation \gls{s2s} est celui de calculer une fonction partielle
\(f: X^\ast \to Y^\ast\), où :
\begin{itemize}
    \item  \(X\) est un ensemble dit d'entrées. 
    \item \(Y\) est un ensemble dit de sorties.
    \item Pour un ensemble \(A\), \(A^\ast = \bigcup\limits_{n\in\naturals}A^n\)
    est l'ensemble de suites de longueur finie d'éléments de \(A\).
\end{itemize}
\(f\) prend donc un \(x = (x_1, x_2,\cdots, x_n)\in X^n\) 
et renvoie un \(y = (y_1, y_2,\cdots, y_m)\in Y^m\).
Dans le cas général, \(n\neq m\) et aucune hypothèse d'alignement n'est supposée.
Il est souvent de prendre \(X = \reals^{d_e}\) et \(Y = \reals^{d_s}\) avec \(d_e, d_s\in\naturals\).
Dans ce cas, 
\(x\in \reals^{d_e \times n}\) et \(y\in \reals^{d_s \times m}\).
Les indices peuvent représenter une succession temporelle 
ou un ordre plus abstrait comme celui des mots dans une phrase~\cite{Martins_2018}.

% todo: add citation
La majorité des outils mathématiques historiquement utilisées pour ce problème 
viennent de la théorie du traitement de signal numérique.
Cependant, l'approche actuellement dominante et celle qui a fait preuve de plus de succès,  
est de les combiner avec les réseaux de neurones.