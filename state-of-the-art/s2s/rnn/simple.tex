\subsection{\Glsfmtlongpl{rnn} simples}

Mathématiquement, une couche d'un \gls{rnn} prend la forme suivante
\begin{equation}
    \label{eq.rnn}
    h^{(t)} = \phi\left(Uh^{(t-1)} + Wx^{(t)} + b\right) \qquad 1 \le t \le n
\end{equation}
où \(t\) est le temps\footnote{Il peut être continu ou discret, réel ou abstrait.},
\(x^{(t)}, h^{(t)}\) sont respectivement l'entrée et la sortie à l'instant \(t\), 
\(n\) est la longueur de la séquence et \(\phi\) est la fonction d'activation~\cite{Fathi_2021}.
Dans le cas où \(\phi\) est l'identité,  
la transformée en \(z\) de l'équation~\ref{eq.rnn} est donnée par
\begin{equation}
    \label{eq.rnn-tz}
    H(z) = z\left(zI - U\right)^{-1} \left(WX(Z) + b\right)
\end{equation}
il s'agit donc d'un système à réponse impulsionnelle infinie~\cite{Fathi_2021}.
Par conséquent, un \gls{rnn} possède une \emph{mémoire}.
Elle émerge du fait qu'il aie ce qu'on appelle \emph{un état interne}.
Dans ce cas, le rôle d'état est joué par les sorties \(h^{(t)}\), qu'on appelle donc états cachés.
