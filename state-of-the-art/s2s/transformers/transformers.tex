\section{Transformeurs}

Les architectures discutées dans les \crefrange{sec. mlp}{sec.cnn}
(notamment les encodeurs-décodeurs à base de \glspl{cnn} et \glspl{rnn}) 
forment la colonne vertébrale des modèles classiques d'apprentissage \gls{s2s}~\cite{deep-nmt-survey}.
Cependant, leur mise en place est inhibée par des problèmes de performance 
(Voire \cref{sec.rnn,sec.cnn}).

Le transformeur (aussi appelé réseau de neurones auto-attentif) est une architecture performante pour le traitement de séquences~\cite{Shim_Sung_2022}.
Introduite par~\cite{attention}, elle est basée sur le mécanisme d'attention~\cite{Larochelle_Hinton_2010}.
Une opération mathématique parallélisable à l'instar du produit de convolution, 
mais dont la complexité ne dépend pas de la distance dans les séquences.



\subimport{}{general}
\subimport{}{embedding}
\subimport{}{attention}
\subimport{}{others}