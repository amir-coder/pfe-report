\subsection{Analyse comparative de la performance}

\begin{table}[htb]
    \centering
    \begin{tabular}{lccc}
        \toprule
        Type du module  
        & Complexité       
        & \stackanchor{Nombre d'Opérations}{Séquentielles} 
        & \stackanchor{Longueur du Chemin}{Emprunté par le Gradient}  \\
        \hline
        Auto-Attention & \(O(n^2 \cdot d)\)         & \(O(1)\) & \(O(1)\)        \\
        Recurrent      & \(O(n \cdot d^2)\)         & \(O(n)\) & \(O(n)\)        \\
        Convolutif     & \(O(k \cdot n \cdot d^2)\) & \(O(1)\) & \(O(log_k(n))\) \\
        \gls{mlp}      & \(O(n \cdot d^2)\)         & \(O(1)\) & \(+\infty\)     \\
        \bottomrule
    \end{tabular}
    \caption[Analyse comparative de la performance]{
      Analyse comparative de la performance des différents types de modèles présentés dans ce chapitre. 
      \(n\) est la longueur de la séquence, 
      \(d\) est la dimension de la représentation vectorielle des éléments 
      et \(k\) est la taille du noyau des convolutions~\cite[Tab. 1]{attention}.
    }
    \label{tab.op_complexities}
\end{table}
