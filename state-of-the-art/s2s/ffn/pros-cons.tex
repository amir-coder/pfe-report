\subsection{Avantages et inconvénients}

Les \glspl{ffn} présentent deux avantages par rapport aux architectures discutées dans le reste de ce chapitre.
Le premier est leur simplicité. 
Elle les rend plus simples à comprendre et à implémenter.
Le deuxième est le fait qu'ils traitent indépendamment les sous-séquences.
Cela rend très facile la tâche de les paralléliser, 
et par conséquent, il accélère considérablement leur entraînement.

Cependant, ce dernier point pose un grand problème.
Comme ils traitent indépendamment les bloques de la séquence, 
les \glspl{ffn} ne peuvent pas modéliser les dépendances inter-bloc.
Par conséquent, leur performance sur les séquences composées de plusieurs blocs est très médiocre.
La solution de ce problème est d'augmenter la taille du bloc (est donc aussi la dimension d'entrée).
Or, cela suppose un alignement par blocs entre les deux séquences.
Une hypothèse invalide selon la Section~\ref{sec.statement}.