\chapter{Apprentissage \glsfmtlong{s2s}}%
\label{chap.s2s}

Les modèles \gls{s2s} sont une famille d'algorithmes d'\gls{ml}
dont l'entrée et la sortie sont des séquences~\cite{Martins_2018}.
Plusieurs tâches de \gls{ml}, notamment en \gls{nlp}, 
peuvent être formulées comme tâches d'apprentissage \gls{s2s}.
Parmi ces tâches, nous citons : la création de chatbots, la réponse aux questions
et --- crucialement pour ce travail --- la \gls{mt} et l'\gls{asr}~\cite{Fathi_2021}.

Dans ce chapitre, nous commençons par formuler le problème de modélisation de séquences.
Ensuite, nous présentons les architectures neuronales les plus utilisées pour cette tâche.
Enfin, nous concluons avec une étude comparative de celles-ci.

\subimport{}{problem}
\subimport{mlp}{mlp}
\subimport{}{encoder-decoder}
\subimport{rnn}{rnn}
\subimport{cnn}{cnn}
\subimport{transformers}{transformers}
\subimport{}{conclusion}
