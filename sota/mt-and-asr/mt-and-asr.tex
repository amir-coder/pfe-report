\chapter{Traduction automatique et reconnaissance automatique de la parole}
\label{chap.mt-and-asr}

Dans le premier chapitre, nous avons introduit la \gls{mt} et l'\gls{asr} 
comme chemins possibles pour la réhabilitation de la parole chez les patients de l'aphasie de Broca.
Ensuite, dans le chapitre précédent, nous avons présenté le problème général
dont ces deux tâches sont des cas particuliers : celui de la modélisation \gls{s2s}.
Nous avons posé formellement le problème 
et présenté les architectures neuronales majeures qui ont été utilisées pour le résoudre en les comparant.
Dans ce chapitre, nous abordons les aspects spécifiques de ces deux tâches
et nous étudions l'application des architectures présentées (notamment le transformeur) dans leur contexte.

\subimport{mt}{machine-translation}
\subimport{asr}{automatic-speech-recognition}
\subimport{}{conclusion}