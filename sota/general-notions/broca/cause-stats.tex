\subsection{Causes, prévalence et incidence}

Les \glspl{avc} sont la première cause d'aphasie~\cite{Hallowell_2017}.
En effet, \(30\%\) des individus atteints d'un \gls{avc} 
développent une aphasie~\cite{Flowers_Skoretz_Silver_Rochon_Fang_Flamand-Roze_Martino_2016}.
Parmi ces individus, \(43\%\) ont une aphasie de Broca~\cite{CNSA_2015}.
Étant donné que 13 millions de personnes sont touchées par un \gls{avc} 
chaque année~\cite{Smaili_Langlois_Pribil_2022},
cela représente environ \(3.9\) millions de cas d'aphasie par an (une incidence d'un peu moins de \(0.05\%\)).
Inversement, \(75\%\) des cas d'aphasie sont causés par un \gls{avc}~\cite{CNSA_2015}.

Il est difficile d'estimer l'incidence et la prévalence globales de l'aphasie.
Ceci est dû au manque de données dans la majorité des pays du monde.
Cependant, le peu de données disponibles donnent des valeurs consistantes avec le \(0.05\%\) estimé ci-dessus.
Selon l'association nationale de l'aphasie~\cite{Home}, 2 millions d'Américains en souffrent, 
soit une prévalence de \(0.6\%\).
En France, ce chiffre est de l'ordre de 300 000 cas, 30 000 desquels sont nouveaux~\cite{CNSA_2015}.
Ceci donne une prévalence de \(0.44\%\) et un taux d'incidence de \(0.044\%\).

L'âge est un facteur de risque très important pour les \gls{avc},
il l'est donc également pour l'aphasie. 
En effet, l'age moyen des individus français atteints de l'aphasie et 73 ans.
\(75\%\) parmi eux sont âgés de plus de 65 ans dont \(25\%\) dépassent les 80 ans~\cite{CNSA_2015}.


% Les \Glsxtrshort{avc} sont la première cause d'aphasie~\cite{Hallowell_2017}. 
% Ils représentent \(75\%\) des cas.
% Le traumatisme crânien en provoque \(5\%\) 
% et les \(20\%\) restants se partagent entre les autres causes.

% Il est difficile d'estimer l'incidence et la prévalence globales de l'aphasie.



% \(33\%\) des \Glsxtrshort{avc} résultent en une aphasie~\cite{CNSA_2015}.
% Un an après l'\Glsxtrshort{avc}, \(13\%\) des patients développent une aphasie de Broca.
