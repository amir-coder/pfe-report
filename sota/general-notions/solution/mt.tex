\section{\Glsfmtlong{mt}}

Nous avons établi que les difficultés d'accès aux traitements pour l'aphasie de Broca 
constituent un risque inadmissible en vue de ses effets potentiellement catastrophiques.
Il est donc urgent de trouver des solutions pour y remédier.
Une solution automatique à base de \gls{mt} semble être une piste intéressante à explorer.
Elle présente la possibilité de corriger les erreurs d'un individu atteint de l'aphasie de Broca
sans avoir besoin d'un orthophoniste.

\subsection{Généralités}

La \gls{mt} est une branche du \gls{nlp}.
Elle étudie l'utilisation des systèmes informatiques pour traduire le texte ou la parole 
d'une langue (appelée source) vers une autre (appelée cible)~\cite{routledge}.
Dans cette section, nous introduisons la \gls{mt} du texte 
pour donner un point de référence aux discussions des chapitres suivants.

\subsection{Classification}

Plusieurs méthodes de \gls{mt} sont présentées dans la littérature.
Ces méthodes peuvent être classées selon les outils mathématiques qu'elles utilisent.
On distingue ainsi trois grandes familles de méthodes de \gls{mt}~\cite{deep-nmt-survey} :
\begin{enumerate}[label=\arabic*. ]
    \item Des méthodes basées sur des connaissances linguistiques (règles de traduction).
    \item Des méthodes basées sur les statistiques.
    \item Des méthodes basées sur les réseaux de neurones.
\end{enumerate}
On les appelle respectivement \gls{rmbt}, \gls{smt}, \gls{nmt}.

À l'intérieur de ces familles, les méthodes peuvent être distinguées en sous-familles.
Ceci donne lieu à la hiérarchie représentée par la Figure~\ref{fig.mt-taxonomy-tree}
\begin{figure}[hbt]
    \centering
    \resizebox{\textwidth}{!}{\begin{tikzpicture}
    \tikzset{every tree node/.style={align=center,anchor=north}}
    \tikzset{level distance=80pt, sibling distance=18pt}
    \Tree 
    [.{Traduction automatique} 
        [.{Aproche rationnelle} 
            [.{Méthodes à base de règle} 
                {Méthodes\\dirèctes} 
                {Méthodes\\de transfert} 
                {Méthodes\\par interlingue} 
            ] 
        ] 
        [.{Aproche empirique} 
            [.{Méthodes statistiques} 
                {À base de\\phrases} 
                {Hierarchique\\à base de\\phrases} 
                {À base de\\syntaxe} 
            ] 
            [.{Méthodes neuronales} 
                {Shallow} %todo translate shallow to french
                {TAS avec un\\modèle de langage\\neuronale} 
                {TAN profonde} 
                {TAN à base de\\mécanismes d'attention} 
            ] 
        ] 
    ]
\end{tikzpicture} }
    \caption[Taxonomie des méthodes de traduction automatique.]
    {Taxonomie des méthodes de traduction automatique~\cite{deep-nmt-survey,hybrid-mt}.}
    \label{fig.mt-taxonomy-tree}
\end{figure}

Une autre classification récurrente dans la littérature 
est celle du triangle de Vauquois (voir Figure~\ref{fig.vauquois-triangle}).
\begin{figure}[hbt]
    \begin{center}
        \begin{tikzpicture}[every edge quotes/.style = {auto, font=\footnotesize, sloped}]
   \node (LS) at (0, 0) {\acrshort{ls}};
   \node (IS) at (1, 1.5) {};
   \node (LC) at (4, 0) {\acrshort{lc}};
   \node (IC) at (3, 1.5) {};
   \node (IL) at (2, 3) {\acrshort{il}};
   \graph {
      (LS) ->["Directe"'] (LC),
      (LS) ->["Analyse"] (IL),
      (IL) ->["Géneration"] (LC),
      (IS) ->["Transfert"'] (IC)
   };
\end{tikzpicture}
    \end{center}
    \caption[Triangle de Vauquois.]
    {Triangle de Vauquois~\cite{hybrid-mt}.}
    \label{fig.vauquois-triangle}
\end{figure}
Elle se base sur l'utilisation ou non de représentations intermédiaires des langues.
Si la phrase traduite est construite directement à partir de la phrase à traduire,
il s'agit de traduction \emph{directe}.
Ce paradigme présente deux inconvénients majeurs.
Le premier --- et le plus évident --- est la difficulté de passer directement d'une langue à une autre.
Le deuxième est la scalabilité : pour maintenir la traduction entre \(n\) langues,
\(\frac{n(n - 1)}{2}\) couples de fonctions de traduction sont nécessaires~\cite{routledge}.

Les méthodes indirectes se distinguent en méthodes ``par transfert'', 
qui utilisent une représentation intermédiaire dépendant de la langue (par exemple, arbre de syntaxe)
et en méthodes ``par interlingue'',
qui utilisent la même représentation intermédiaire pour toutes les langues.
Ce dernier paradigme résout le problème de scalabilité 
(seulement \(n\) couples de traductions sont nécessaires)
au coût de construire une interlingue assez riche à représenter toute phrase dans toute langue~\cite{routledge}.

Dans ce travail, 
nous nous intéressons principalement aux sous arbre droit de la Figure~\ref{fig.mt-taxonomy-tree}.
En effet, la majorité de notre investigation porte sur son dernier élément (la \gls{nmt} profonde).
Ceci est motivé par l'énorme succès 
dont l'\gls{dl} a fait preuve dans la dernière décennie~\cite{Raschka_Mirjalili_2017}.
