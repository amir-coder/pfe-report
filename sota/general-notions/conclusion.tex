\section{Conclusion}%
\label{sec.general_notions.conclusion}

Dans ce chapitre, nous avons fourni une brève introduction aux sujets abordés dans ce mémoire.
Nous avons introduit le problème central, ses spécificités, son contexte historique et scientifique 
et des chemins possibles vers une solution.

Nous avons commencé par présenter au lecteur les troubles linguistiques qui sont 
l'aphasie en général et l'aphasie de Broca en particulier.
Nous l'avons abordé sous différents angles, notamment celui de l'histoire et de la neuroanatomie.
Nous l'avons également mis dans le contexte des autres syndromes aphasiques.
Une discussion des causes de l'aphasie de Broca a aussi été menée.
Des statistiques sur l'incidence et la prévalence de l'aphasie de Broca,
ainsi que des études sur ses conséquences ont été présentées
pour aider le lecteur à apprécier l'ampleur du problème et sa gravité.

Nous avons terminé notre introduction à l'aphasie de Broca 
par une discussion des traitements typiquement déployés contre elle.
En particulier, nous avons discuté le manque d'accessibilité et de scalabilité du traitement orthophonique.
Un problème que nous avons jugé très grave à cause des conséquences dévastatrices de l'aphasie de Broca.

Après cela, nous avons introduit les techniques de \gls{nlp} que nous jugeons pertinentes
pour la résolution de notre problème : la \gls{mt} et l'\gls{asr}.
Pour la \gls{mt}, nous avons commencé par une description de la tâche.
Ensuite, nous avons avancé les classifications de ses méthodes qu'on trouve dans la littérature.
Notre choix d'exploration s'est porté sur la classe que nous avons trouvée la plus prometteuse~:
la \gls{nmt}.

Pour l'\gls{asr}, nous avons repris la même démarche.
Après avoir décrit la tâche, nous avons illustré une classification des méthodes.
Nous avons encore une fois choisit de nous concentrer sur les techniques à base de \gls{dl}.

Dans la suite de ce document, nous étudions plus en détail la \gls{nmt} et l'\gls{asr}.
Le chapitre suivant aborde l'apprentissage \gls{s2s}, 
le problème général dont ces deux tâches sont des sous-problèmes.
Celui qui suit rentre dans le détail de l'application de la \gls{nmt} et de l'\gls{asr}.
